\documentclass[../main.tex]{subfiles}
\begin{document}
%- experiences ------------------------------------------------------------%
  \pdfbookmark{Experience}{Experience}
  \begin{category}
    \section{Experience}
    \citembullet{
      \textbf{Founder} of \textbf{CoachCrew.tech}
      \strut\hfill August 2023 -- Present\\[-9pt]
    }
    \citembullet{
      \textbf{Software developer} at \textbf{Deutsche Börse} 
      \strut\hfill February 2023 -- July 2023 \\[-9pt]
    }
      \begin{itemize}
        \item \emph{\textbf{Project:} Cloud Stream}
        Developed and maintained an optimized client
        to capture messages at the highest rate and conduct stress 
        test to the framework. 
        Improved the existing architecture by introducing static images 
        created via Packer, and helped the team 
        in the migration of the product from Amazon Web Services to Google 
        Cloud Platform.
      \end{itemize}

    \citembullet{
      \textbf{Research assistant} at 
      \textbf{Max Planck Insitute for Software Systems} 
      under the supervision of 
      \textbf{\href{https://people.mpi-sws.org/~antoinek}
      {Prof. Antoine Kaufmann}}
    } \strut\hfill October 2019 -- January 2023\\[-9pt]
      \begin{itemize}
        \item \emph{\textbf{Project:} 
          TCP Acceleration as a Service in Virtualized Environments
        }
        In this project, I explored the needs for network 
        virtualization in public cloud environment. The 
        goal of the project is to enable modern, CPU efficient 
        network stacks for cloud tenants while enforcing 
        isolation. To this end, I added a netdev driver to 
        Open vSwitch code and used TAS as an efficient
        packet processing stack for TCP.

        \item \emph{\textbf{Project:} 
          Exploring Domain-Specific Architectures 
          for Network Protocol Processing
        } 

        As part of this project, I worked with 
        Xilinx UltraScale+ FPGA-based NICs. I implemented a 
        userspace driver for the PCIe communication using 
        vfio driver.

      \end{itemize}

    \citembullet{
      \textbf{Research Intern} at \textbf{SAFARI Group} 
      headed by \textbf{
        \href{https://people.inf.ethz.ch/omutlu/}{Prof. Onur Mutlu}
      }, ETH  \hfill July - September 2018
      \begin{itemize}
        \item \emph{\textbf{Project:} 
	      Towards Practical, Efficient, and Realizable 
        Hardware-Software Interfaces to Enhance Application 
        Expressivity
      } 

	    This work investigates the possibility of designing rich 
      hardware-software interfaces. As part of the project,
      I added custom instructions to RISC-V, and implemented
      a custom module on Rocket Chip which 
      can be used by programmers to convey high-level data 
      structures to the underlying hardware. 
        

    \end{itemize}
    }

    \citembullet{
      \textbf{Undergraduate Research Assistant} 
      at \textbf{
        \href{http://dsn.ce.sharif.edu/}
        {Data Storage, Networks, \& Processing}
      }
      Laboratory headed by 
      \textbf{
        \href{http://sharif.edu/~asadi}{Prof. Hossein Asadi}
      }, Sharif University of Technology  \hfill 
      September 2016 - April 2018 \\[-9pt]
    \begin{itemize}
       \item \emph{\textbf{Project:} SSD Reliability Under Adverse Condition }
        This project focuses on the behavior of Solid-State Disks 
        (SSDs) under power fault. I made an automatic failure 
        testing framework, which consists of hardware and software 
        sections. By applying the testing framework, we tested more 
        than 5 SSDs from different vendors and detected different 
        types of data failures and device failure.

    \end{itemize}
  }
\end{category}
\end{document}
